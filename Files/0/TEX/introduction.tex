\documentclass{letter}
\pagestyle{empty}
%\title{Introduction}
\begin{document}
\begin{center}
Introduction
\end{center}

\newcommand{\en}{\phantom{N}}
%\maketitle
\noindent
%%%%%%%%%%%%%%%%%%%%%%%%%%%%%%%%%%%%%%%%%%%%%%%%%%%%%%%%%%%%%%%%
\textsc{In this collection of papers},
we begin to develop a GIS system that is is more comprehensible
and more accessible than what presently dominates the industry.

%%%%%%%%%%%%%%%%%%%%%%%%%%%%%%%%%%%%%%%%%%%%%%%%%%%%%%%%%%%%%%%%
\textsc{One goal}
is that a wide range of GIS applications---with
emphasis on computational geometry---be
comprehensible to the very young.
Young people should have tools to \emph{precisely quantify}
concepts such as ``near,'' ``within,'' ``in the direction of,''
and so forth.

%%%%%%%%%%%%%%%%%%%%%%%%%%%%%%%%%%%%%%%%%%%%%%%%%%%%%%%%%%%%%%%%
\textsc{Although conventional}
geometry and trigonometry, as taught in many school systems,
take a steps in this direction,
they go off on a tangent whith respect to
the \emph{central} question of GIS:
\begin{center}
Where?
\end{center}

%%%%%%%%%%%%%%%%%%%%%%%%%%%%%%%%%%%%%%%%%%%%%%%%%%%%%%%%%%%%%%%%
\textsc{The number system}, we believe, is the natural starting point
for a journey toward a more comprehensible, more accessible GIS.
Numbers and quantity go hand in hand.
To the extent we are able, we will favor
\begin{itemize}
\item integers rather than fractions or real numbers,
\item multiplication rather than division, and
\item squares rather than square roots
\end{itemize}

\textsc{We acknowledge} N J Wildberger for a great deal of inspiration as well as content.
Much of what is presented herein flows from his work,
especially his 2005 book
\textit{Divine Proportions: Rational Trigonometry to Universal Geometry}
(Wild Egg Pty Ltd; Australia. ISBN 097574920X).
The author also has gained much from Wilberger's numerous
lectures (search: ``wildtrig'') on various topics in pure and
applied mathematics.
\end{document}