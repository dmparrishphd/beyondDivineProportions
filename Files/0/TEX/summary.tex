\documentclass{letter}
\pagestyle{empty}
\begin{document}
\begin{center}
Summary
\end{center}

This collection of articles elaborates on the consequences of a
design choice in which geometric objects are built up without
using floating-point arithmetic, but with integers only.

We begin with ``Motivation,'' whose title is sufficient to
summarize its content.

The main arc of the collection is to establish storage
requirements (i.e., how many bits and words are needed) in order
to achieve or surpass the precision many GIS professionals have
grown accustomed to.

In ``Storage Requrements,'' we show that 64-bit integers are
more than sufficient to surpass the precision of conventional
typical floating-point point numbers. Additionally, they can
be used efficiently to build up more complex structures, such as
lines.

So as not interrupt the flow of ``Storage Requirements,'' the
subtopics of ``Line Amid Two Points,'' ``Line through a Point
and the Centroid of a P-Parallelogram'' (a p-parallelogram is a
special kind of parallelogram), and ``Line through the Centroids
of Two P-Parallelograms'' are presented in separate articles.
\end{document}