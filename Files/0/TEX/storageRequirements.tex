\documentclass{article}
\title{Storage Requirements}
\begin{document}
\newcommand{\en}{\phantom{N}}
\maketitle
\noindent
\textsc{This article}
establishes storage requirements for the structures
\begin{itemize}
\item point
\item line
\item quadrance\footnote{Quadrance can be thought of as the square of distance.} between points
\end{itemize}
\section{Notation}
\subsection{The Symbol $N$}
Unless otherwise specified, the symbol $N$ represents an
arbitrary---constant---power of two.
A secondary aim of this article is to establish additional bands,
if not constraints, on $N$.

\subsection{The Symbols $a$--$z$}
Unless otherwise specified, the symbols
$$a\ b\ c\ d\ e\ f\ g\ h\ i\ j\ k\ l\ m\ n\ o\ p\ q\ r\ s\ t\ u\ v\ w\ x\ y\ z$$
each represent an arbitrary integer.

\subsection{Concatenation and the Symbol $\times$}
$$ab \equiv a\times b$$
That is,
the expression $ab$,
where the symbols $a$ and $b$ are concatenated,
has the same meaning as the expression $a\times b$.

If both $a$ and $b$ are integers, $ab$ is the product of $a$ and $b$.

For example, if $a=6$ and $b=7$, $ab=a\times b=6\times 7=42$.
\subsection{Whitespace and Lists}
Two or more operands separated by whitespace form a list.
The expression
$$a\ b$$
denotes a list containing the items $a$ and $b$---in that order.
Blank lines distingish the above list from what is not the list.

Lists may be preceeded by an open parenthesis and proceeded by
a close parenthesis, indicating the boundaries of the list.
For example,
$$()\ (a)\ (a\ b)$$
denotes a list of three items: an empty list, a list of one item, and list of two items.
\section{Cardinality of a List}
The cardinality of a list is the number of items in that list.
The cardinality operator is $\#$.
For example:
$$\#()=0$$
$$\#(0\ 0\ 0\ 0\ 0\ 0\ 0\ 0\ 0)=9$$
$$\#(f\ o\ u\ r)=4$$

\section{List Items}

Let $l$ be a list.
Provided that $0<k\le \#l$,
$l_k$ is the $k$\textsuperscript{th} item of $l$.
For example, $(11\ 12\ 13)_2=12$.

\section{Types of Lists}

A list may be declared to be of a certain type using a ``type name.''
Operations on particular types of lists may be defined.

\section{Vectors}

Let ``vector'' be a type name. Let
$$\mathrm{vec} = \mathrm{vector}$$
then the expression
$$\mathrm{vec}(a\ b)$$

\noindent
denotes a vector (i.e., a list of type vector).

\section{Untyped Lists}

We might wish to define a list in terms of a typed list:
$$\mathrm{list}\ t(a\ b) = (a\ b)$$

\noindent
where $t$ is a type name. For example,
$$\mathrm{list}\ \mathrm{vec} (a\ b) = (a\ b)$$

\section{Max and Mix}

%%%%%%%%%%%%%%%%%%%%%%%%%%%%%%%%%%%%%%%%%%%%%%%%%%%%%%%%%%%%%%%%
The ``max'' of a list of integers is
an integer among the list items whose value is
not less than any other integer in the list.
Similarly,
The ``min'' of a list of integers is
an integer among the list items whose value is
not greater than any other integer in the list.
For example:
$$\mathrm{max}(0\ 1)=1$$
$$\mathrm{max}(0\ 0\ 0\ 0\ 0\ 0\ 0\ 0\ 0)=0$$
$$\mathrm{min}(0\ 1)=0$$
$$\mathrm{min}(0\ 0\ 0\ 0\ 0\ 0\ 0\ 0\ 0)=0$$

\section{Dot Product of Two Vectors}

Let the symbol $\cdot$ denote the binary operation ``dot product.''
$$\mathrm{vec}\ u \cdot \mathrm{vec}\ v
\equiv
\sum^n_{i=1}
u_i v_i$$
where $u$ and $v$ are lists of integers, $0<\#u=\#v=n$.
For example,
$$\mathrm{vec}\ (2\ 3)\cdot\mathrm{vec}\ (5\ 7)=2\times 5+3\times 7=10+21=31$$

\section{Intervals}

Let ``interval'' be a type name.
Let $\mathrm{itv}=\mathrm{interval}$. Provided that $a\le b$,
$$k\in \mathrm{itv}\ (a\ b) \Leftrightarrow a \le k \le b$$
$$n\not\in \mathrm{itv}\ (a\ b) \Leftrightarrow n<a \vee b<n$$

\noindent
In other words, $\mathrm{itv}\ (a\ b)$ indicates an arbitrary integer $k$,
between $a$ and $b$ inclusive.
And $n$ represents an arbitrary integer
beyond $\mathrm{itv}\ (a\ b)$.

For example,
$$\mathrm{itv}\ (0\ 1)$$
represents an unknown integer that could be either $0$ or $1$,
and that can be only $0$ or $1$.

\subsection{Intervals in Intervals}

$$\mathrm{itv}\ (a\ b)\in \mathrm{itv}\ (c\ d)\Leftrightarrow a\in \mathrm{itv}\ (c\ d)\wedge b\in \mathrm{itv}\ (c\ d)$$
In other words, $\mathrm{itv}\ (a\ b)$ is in $\mathrm{itv}\ (c\ d)$
if and only if
$c\le a\le d$ and
$c\le b\le d$.
For example,
$$\mathrm{itv}\ (1-2^{60}\en 2^{60}-1)\in \mathrm{itv}\ (1-2^{61}\en 2^{61}-1)$$

\subsection{Special Intervals}

\newcommand{\NO}{\mathbf{N_0}}
\newcommand{\NI}{\mathbf{N_1}}
\newcommand{\NII}{\mathbf{N_2}}
\newcommand{\NIII}{\mathbf{N_3}}
\newcommand{\NK}{\mathbf{N_k}}

\newcommand{\NN}{\mathbf{N_0^2}}
\newcommand{\NNI}{\mathbf{N_1^2}}
\newcommand{\NNII}{\mathbf{N_2^2}}
\newcommand{\NNK}{\mathbf{N_k^2}}

$$\NO\equiv \mathrm{itv}(1-N\en N-1)$$
$$\NI\equiv \mathrm{itv}(1-2N\en 2N-1)$$
$$\NII\equiv \mathrm{itv}(1-4N\en 4N-1)$$
$$\NK\equiv \mathrm{itv}(1-2^kN\en 2^kN-1)$$

$$\NN\equiv \mathrm{itv}(1-N^2\en N^2-1)$$
$$\NNI\equiv \mathrm{itv}(1-2N^2\en 2N^2-1)$$
$$\NNII\equiv \mathrm{itv}(1-4N^2\en 4N^2-1)$$
$$\NNK\equiv \mathrm{itv}(1-2^kN^2\en 2^kN^2-1)$$

\subsection{Interval Addition}

$$\mathrm{itv}(a\ b) + \mathrm{itv}(c\ d) \equiv
\mathrm{itv}(a+c\phantom{N}b+d)$$
\noindent
In particular,
$$\NO+\NO\in\NI$$
since
$$\NO+\NO=\mathrm{itv}(1-N\en N-1) + \mathrm{itv}(1-N\en N-1)$$
$$=$$
$$\mathrm{itv}(2-2N\en 2N-2)\in
\mathbf{N_1}$$

\subsection{Interval Subtraction}

$$\mathrm{itv}(a\ b) - \mathrm{itv}(c\ d) \equiv
\mathrm{itv}(a-d\en b-c)$$
In particular,
$$\NO-\NO\in\NI$$
since
$$\NO-\NO=\mathrm{itv}(1-N\en N-1) - \mathrm{itv}(1-N\en N-1)$$
$$=$$
$$\mathrm{itv}(2-2N\en 2N-2)\in
\mathbf{N_1}$$

\subsection{Interval Multiplication}

$$\mathrm{itv}(a\ b) \times \mathrm{itv}(c\ d) \equiv
\mathrm{itv}(\mathrm{min}\ l\en \mathrm{max}\ l)$$
where $l=(ac\ ad\ bc\ bd)$.
In particular,
$$\NO\times\NO\in\NN$$
since
$$\NO\times\NO=\mathrm{itv}(1-N\en N-1) \times \mathrm{itv}(1-N\en N-1)$$
$$=$$
$$\mathrm{itv}(-N^2+2N-1\en N^2-2N+1)\in
\mathrm{itv}(1-N^2\en N^2-1)\in\NN$$

\section{Points}

Let ``point'' be a type name.

Unless otherwise specified,
$$\#\ \mathrm{list}\ p = 2$$
where $p$ is a point. I.e., points are two-dimensional.

For $\mathrm{point}(a\ b)$,

$$a\in \NO, b\in \NO$$

\noindent
In other words, each coordinate value,
$a$ and $b$ is taken from a finite range of integers between
$1-N$ and $N-1$.

\section{Definition of Vector Quadrance}

Let $\mathrm{q} = \mathrm{quadrance}$, a unary operator.
The quadrance of a vector $v$ of integers is the integer:
$$\mathrm{q}v\equiv v\cdot v$$

\noindent
In particular,
$\mathrm{q}\ \mathrm{vec}(a\ b)=
\mathrm{vec}(a\ b)\cdot
\mathrm{vec}(a\ b)$.
And, provided that $a, b\in\NO$:
$$\mathrm{q\ vec(a\ b)}\in \NO\NO+\NO\NO\in \NN+\NN\in \NNI$$

\section{Proportions}

Let ``proportion'' be a type name.
Let $\mathrm{prp} = \mathrm{proportion}$.
Provided that $l$ is a list of integers;
$1<\#l$,
$\mathrm{q\ vec}\ l \ne 0$:
$$\mathrm{prp}\ l$$
is a proportion.
An expression equivalent to $\mathrm{prp}(0, 0)$ is undefined.
However, both
$\mathrm{prp}(1, 0)$ and $\mathrm{prp}(0, 1)$
are defined.

\subsection{Equivalent Proportions}

$$\mathrm{prp}(a\ b) = \mathrm{prp}(c\ d) \Leftrightarrow ad = bc$$

\section{Vector Difference}

Provided that $u$ anv $v$ are vectors of integers, $0<\#u=\#v=n$,
$$u-v\equiv \mathrm{vec}(u_1 - v_1\en u_2 - v_2\en \ldots\en u_n - v_n)$$
Provided that $u_k\in\NO$ and $v_k\in\NO$,
$w_k\in\NI$,
where $w=u-v$.

\section{Vector Sum}

Provided that $u$ anv $v$ are vectors of integers, $0<\#u=\#v=n$,
$$u+v\equiv \mathrm{vec}(u_1 + v_1\en u_2 + v_2\en \ldots\en u_n + v_n)$$
Provided that $u_k\in\NO$ and $v_k\in\NO$,
$w_k\in\NI$,
where $w=u+v$.

\section{Differecnce between Points}

$$\mathrm{point}(a\ b) - \mathrm{point}(c\ d)\equiv
\mathrm{vec}(a - c\en b - d)$$
Provided that $a, b, c, d\in \NO$; $a-c\in\NI$, $b-d\in\NI$.

Again, for emphasis, the difference between points is a vector, not a point.

\section{Sum of Two Points}
$$\mathrm{point}(a\ b) + \mathrm{point}(c\ d)\equiv
\mathrm{vec}(a + c\en b + d)$$
Provided that $a, b, c, d\in \NO$; $a+c\in\NI$, $b+d\in\NI$.

Again, for emphasis, the sum of points is a vector, not a point.

\section{Array}
Let ``array'' be a type name.
Let $\mathrm{aa}=\mathrm{array}$\footnote{Mnemonic: ``aa'' is an array of a's.}.

\subsection{Array Product}
Let $u$ and $v$ be arrays, $0<\#u=\#v=n$.
$$w=uv\Leftrightarrow w_i = u_iv_i, \forall\ 1\le i\le n$$

\section{Pitch}

Let ``pitch'' be a type name.

\subsection{Pitch of two Points}

Let $p$ and $r$ be points.
$$\mathrm{pitch}(p\ r)\equiv \mathrm{pitch}(r_1-p_1\en p_2-r_2)$$
In particular, $\mathrm{pitch}(0\ 1)$ is defined, whereas the hypothetical ``slope'' $\frac{1}{0}$ is not.

Also, in particular,
provided that $p_1, p_2, r_1, r_2\in \NO$;
$m=\mathrm{pitch}(p\ r)$:
$m_1, m_2\in \NI$.

\subsection{Pitch-Quadrance}

Let $p$ be a pitch. Then

$$\mathrm{q}\ p\equiv \mathrm{q\ vec\ list}\ p$$

\section{Distinct Points}

Two points $p$ and $r$ are distinct if and only if
$$\mathrm{q}\ s\ne 0$$
where $s=r-p$.

\section{P-Parallelograms}

Let ``p-parallelogram'' be a type name.
Let ppara $=$ p-parallelogram.
Let $p$, $r$, $s$, and $t$ be points distinct from one another.
Provided that
$$\mathrm{pitch}(p\ r) = \mathrm{pitch}(t\ s)\ \mathrm{and}$$
$$\mathrm{pitch}(r\ s) = \mathrm{pitch}(p\ t)$$
$\mathrm{pppara}(p\ r\ s\ t)$ is defined.

\section{Quadrance between Two Points}

Let $p$ and $r$ be points. Let\footnote{Recall that $d$ is a vector.} $d = p - r$.
$$\mathrm{q} (p\ r) \equiv \mathrm{q}\ d$$

Provided that $p, r, s, t\in \NO: d\in\NI, \mathrm{q}\ d\in\NNII$

\section{Lines and Half-Planes I}

Let ``line,'' ``nplane,'' (inclusive half-plane) and ``xplane'' (exclusive half plane) be type names.

Let $l=(c\ a\ b)$, $0<a^2+b^2$.

\subsection{Pitch of Lines and Half-planes}

$$\mathrm{pitch\ line}\ l\equiv \mathrm{pitch}(a\ b)$$
$$\mathrm{pitch\ nplane}\ l\equiv \mathrm{pitch}(a\ b)$$
$$\mathrm{pitch\ xplane}\ l\equiv \mathrm{pitch}(a\ b)$$

\subsection{Dot Product of Vector with Line or Half-Plane}

Let $v$ be a vector, $\#v=3$.
Let $h$ be any one of $\mathrm{line}\ l$, $\mathrm{nplane}\ l$, or $\mathrm{xplane}\ l$.
$$v\cdot h\equiv v\cdot \mathrm{vec\ list}\ h$$

\subsection{Point on a Line}

$$\mathrm{point}(x\ y)\in \mathrm{line}\ l\Leftrightarrow u\cdot l=0$$
where
$$u=\mathrm{vec}(1\ x\ y)$$

\subsection{Point in a Half-Plane}

$$\mathrm{point}(x\ y)\in \mathrm{xplane}\ l\Leftrightarrow u\cdot l  < 0$$
$$\mathrm{point}(x\ y)\in \mathrm{nplane}\ l\Leftrightarrow u\cdot l\le 0$$
where
$$u=\mathrm{vec}(1\ x\ y)$$

\section{Product of a Scalar and a Point}

$$k\times \mathrm{point}(a\ b)\equiv \mathrm{vec}(ka\en kb)$$

\section{Protocentroid of a P-Parallelogram}

$$\mathrm{protocentroid\ ppara}(p\ r\ s\ t)\equiv p+s=r+t,$$
where $p$, $r$, $s$, and $t$ are points.\footnote{Recall
that the sum of two points is a vector.}

Provided that
$p_1, s_1\in\NO;$
$(p + s)_1\in\NI$.

\section{Determinant of Two Points}

Let ``determinant'' be a type name.
Let $\mathrm{det}=\mathrm{determinant}$.
Let $p=(a\ b)$ and $r=(c\ d)$. Then
$$\mathrm{det}(p\ r)\equiv \mathrm{det}(ad-bc).$$
Provided that $a, b, c, d\in\NO;
ad\in\NN, bc\in\NN;
\mathrm{det}(p\ r)\in\NNI$.

\section{Line through Two Points}

Let $p$ and $r$ be distinct points.
Let $l$ be the line passing through $p$ and $r$.
Then
$$l=\mathrm{line}(p_1r_2-p_2r_1\en p_2-r_2\en r_1-p_1)=$$
$$=\mathrm{line}(\mathrm{det}(\mathrm{list}\ p\en \mathrm{list}\ r)\en p_2-r_2\en r_1-p_1)$$

Provided that $p_1, p_2, r_1, r_2\in\NO,
l_1\in\NNI, l_2\in\NI, l_3\in\NI$.

\section{Line Amid Two P-Parallelograms}
(See an accompanying paper for the definition of ``amid.'')

Provided that $x$ and $y$ are lists of eight integers each and
satisfy
$$P\equiv \mathrm{ppara}(
\mathrm{point}(x_1\en y_1)\
\mathrm{point}(x_2\en y_2)\
\mathrm{point}(x_3\en y_3)\
\mathrm{point}(x_4\en y_4))$$
and
$$R\equiv \mathrm{ppara}(
\mathrm{point}(x_5\en y_5)\
\mathrm{point}(x_6\en y_6)\
\mathrm{point}(x_7\en y_7)\
\mathrm{point}(x_8\en y_8))$$
The line $l$ amid each pair of diagonal-points of two p-parallelograms is
%$$l=
%\mathrm{line}(
%x_{13}y_{57}-x_{57}y_{13}\en
%2y_{13}-2y_{57}\en
%2x_{57}-2y_{13})$$
%where $x_{13}=x_1+x_3, y_{13}=y_1+y_3, x_{57}=x_5+x_7, y_{57}=y_5+y_7$.

$$l=
\mathrm{line}(
ad-bc\en
2b-2d\en
2c-2a)$$
where $a=x_1+x_3, b=y_1+y_3, c=x_5+x_7, d=y_5+y_7$.

Provided that $x_k, y_k\in\NO$:
$$l_1\in \NI\NI-\NI\NI\in \NNII-\NNII\in\mathbf{N_3^2}$$
$$l_2, l_3\in 2\NI-2\NI\in \NII-\NII\in \NIII$$

\section{Storage Requirements}
The above considerations tolerate
points with items $\in\NO$ and
lines with zero-order terms $\in\NIII$ and
first-order terms $\in\mathbf{N_3^2}$.

Suppose our computing machinery accomodates words of
$n+4$ bits\footnote{perhaps 64 bits per word},
in which we store any value $\in\NIII$.
Then\footnote{we might choose $n=60$.}
$n$ corresponds with values in $\NO$.
Two such words would have $2n+8$ bits,
which is enough to store any value $\in\mathbf{N_8^2}$ and therefore
any value $\in\mathbf{N_3^2}$.
\end{document}
