\documentclass{article}
\title{Line through\\ the Centroids of\\ Two P-Parallelograms}
\begin{document}
%%%%%%%%%%%%%%%%%%%%%%%%%%%%%%%%%%%%%%%%%%%%%%%%%%%%%%%%%%%%%%%%
\maketitle
\newcommand{\en}{\phantom{N}}
\noindent
Let
$P = \mathrm{ppara}(
\mathrm{point}(x_1\en y_1)\
\mathrm{point}(x_2\en y_2)\
\mathrm{point}(x_3\en y_3)\
\mathrm{point}(x_4\en y_4))$.
Let\footnote{Recall
that the centroid of a parallelogram is the midpoint of either diagonal.}
\[
2C_P =
\mathrm{vec}(x_{13}\en y_{13}) = 
\mathrm{vec}(x_{24}\en y_{24}).
\]
where
$x_{13} \equiv x_1 + x_3,\ y_{13} \equiv y_1 + y_3,\
x_{24} \equiv x_2 + x_4,\ y_{24} \equiv y_2 + y_4$.
Similarly, let
$R = \mathrm{ppara}(
\mathrm{point}(x_5\en y_5)\
\mathrm{point}(x_6\en y_6)\
\mathrm{point}(x_7\en y_7)\
\mathrm{point}(x_8\en y_8))$.
Let
\[
2C_R =
\mathrm{vec}(x_{57}\en y_{57}) =
\mathrm{vec}(x_{68}\en y_{68}).
\]
where
$x_{57} \equiv x_5 + x_7,\ y_{57} \equiv y_5 + y_7,\
x_{68} \equiv x_6 + x_8,\ y_{68} \equiv y_6 + y_8$.

Provided that $C_P \ne C_R$,
the line through the centroids of $P$ and $R$ satisfies
$$\mathrm{vec}(
x_{13}y_{57}-x_{57}y_{13}\en
2y_{13}-2y_{57}\en
2x_{57}-2x_{13})
\cdot
\mathrm{vec}(1\ x\ y)
$$
\end{document}
